\chapter{数据结构}

\section{ST表}

\begin{lstlisting}
const int N = 5e4 + 10;
int big[N][30], small[N][30];

// i_j => [i, i + 2^j - 1]

int n, q;
int a[N], pre[N];

int main() {
  scanf("%d %d", &n, &q);
  for (int i = 1; i <= n; ++i) scanf("%d", a + i), pre[i] = pre[i / 2] + 1;
  for (int i = 1; i <= n; ++i)
    big[i][0] = small[i][0] = a[i];
  for (int i = 1; i <= pre[n]; ++i)
    for (int j = 1; j + (1 << i) - 1 <= n; ++j)
      big[j][i] = max(big[j][i - 1], big[j + (1 << (i - 1))][i - 1]),
      small[j][i] = min(small[j][i - 1], small[j + (1 << (i - 1))][i - 1]);
  while (q --) {
    int l, r, len;
    scanf("%d %d", &l, &r);
    len = pre[r - l + 1] - 1;
    printf("%d\n", max(big[l][len], big[r - (1 << len) + 1][len]) - min(small[l][len], small[r - (1 << len) + 1][len]));
  }
}

\end{lstlisting}


\section{树状数组}

\begin{lstlisting}
inline void add(vector<int> &c, int pos, int sz, int val) {
  for (int x = pos; x <= sz; x += x & -x)
    c[x] += val;
}

inline int query(vector<int> &c, int pos) {
  int res = 0;
  for (int x = pos; x; x -= x & -x)
    res += c[x];
  return res;
}
\end{lstlisting}

\section{线段树}

易错:

\begin{enumerate}
  \item tag 在pushdown后没有清空
  \item 线段树开小了导致越界
\end{enumerate}

\begin{lstlisting}
/*
 * 区间加法,区间求和
 */
#include <cstdio>
#include <vector>
#include <iostream>
#include <algorithm>
using namespace std;

const int N = 1e5 + 10;
#define int long long
int n, m;

struct Seg {
  int l, r;
  int sum, add;
} t[N << 2];

int a[N];

#define l(o) t[(o)].l
#define r(o) t[(o)].r
#define sum(o) t[(o)].sum
#define add(o) t[(o)].add
#define cint const int&

inline void build(int o, int l, int r) {
  l(o) = l, r(o) = r;
  if (l == r) {
    sum(o) = a[l]; return;
  }
  int mid = l + ((r - l) >> 1);
  build(o << 1, l, mid), build(o << 1 | 1, mid + 1, r);
  sum(o) = sum(o << 1) + sum(o << 1 | 1);
}

inline void pushdown(int o) {
  if (add(o)) {
    sum(o << 1) += 1ll * (r(o << 1) - l(o << 1) + 1) * add(o);
    sum(o << 1 | 1) += 1ll * (r(o << 1 | 1) - l(o << 1 | 1) + 1) * add(o);
    add(o << 1) += add(o); add(o << 1 | 1) += add(o);
    add(o) = 0;
  }
}

inline void update(int o, cint l, cint r, cint val) {
  if (l <= l(o) && r(o) <= r) {
    add(o) += val; sum(o) += 1ll * (r(o) - l(o) + 1) * val;
    return;
  }
  pushdown(o);
  int mid = l(o) + ((r(o) - l(o)) >> 1);
  if (l <= mid) update(o << 1, l, r, val);
  if (r > mid) update(o << 1 | 1, l, r, val);
  sum(o) = sum(o << 1) + sum(o << 1 | 1);
}

int query(int o, int l, int r) {
  if (l <= l(o) && r(o) <= r) return sum(o);
  pushdown(o);
  int res = 0;
  int mid = l(o) + ((r(o) - l(o)) >> 1);
  if (l <= mid) res += query(o << 1, l, r);
  if (r > mid) res += query(o << 1 | 1, l, r);
  return res;
}

signed main() {
  scanf("%lld %lld", &n, &m);
  for (int i = 1; i <= n; ++ i) scanf("%lld", a + i);
  build(1, 1, n);
  char s[10];
  while (m --) {
    int x, y, z;
    scanf("%s", s + 1);
    if (s[1] == '2') {
      scanf("%lld %lld", &x, &y);
      printf("%lld\n", query(1, x, y));
    } else {
      scanf("%lld %lld %lld", &x, &y, &z);
      update(1, x, y, z);
    }
  }
}

/*
 * Prob:{l, r} 是否完全覆盖询问
 * 离散化前添加节点 l+1, r+1,防止:
 *     2     4
 *   1 2  ^  4 5
 */
#include <cstdio>
#include <vector>
#include <iostream>
#include <algorithm>
using namespace std;

const int N = 1e5 + 10;
#define int long long
int n, m;

struct Seg {
  int l, r;
  int sum, add;
}t[N << 2];

int ql[N], qr[N];
int sorted[N];

#define l(o) t[(o)].l
#define r(o) t[(o)].r
#define sum(o) t[(o)].sum
#define add(o) t[(o)].add
#define cint const int&

inline void build(int o, int l, int r) {
  l(o) = l, r(o) = r; add(o) = 0;
  if (l == r) {
    sum(o) = 0; return;
  }
  int mid = l + ((r - l) >> 1);
  build(o << 1, l, mid), build(o << 1 | 1, mid + 1, r);
  sum(o) = sum(o << 1) + sum(o << 1 | 1);
}

inline void pushdown(int o) {
  if (add(o)) {
    sum(o << 1) = 1ll * (r(o << 1) - l(o << 1) + 1) * add(o);
    sum(o << 1 | 1) = 1ll * (r(o << 1 | 1) - l(o << 1 | 1) + 1) * add(o);
    add(o << 1) = add(o); add(o << 1 | 1) = add(o);
    add(o) = 0;
  }
}

inline void update(int o, cint l, cint r, cint val) {
  if (l <= l(o) && r(o) <= r) {
    add(o) = val; sum(o) = 1ll * (r(o) - l(o) + 1) * val;
    return;
  }
  pushdown(o);
  int mid = l(o) + ((r(o) - l(o)) >> 1);
  if (l <= mid) update(o << 1, l, r, val);
  if (r > mid) update(o << 1 | 1, l, r, val);
  sum(o) = sum(o << 1) + sum(o << 1 | 1);
}

int query(int o, int l, int r) {
  if (l <= l(o) && r(o) <= r) return sum(o);
  pushdown(o);
  int res = 0;
  int mid = l(o) + ((r(o) - l(o)) >> 1);
  if (l <= mid) res += query(o << 1, l, r);
  if (r > mid) res += query(o << 1 | 1, l, r);
  return res;
}

signed main() {
  int t; scanf("%lld", &t);
  while (t --) {
    scanf("%lld", &n); m = 0;
    for (int i = 1; i <= n; ++i) {
      scanf("%lld %lld", ql + i, qr + i);
      sorted[++m] = ql[i];
      sorted[++m] = ql[i] + 1; // important
      sorted[++m] = qr[i];
      sorted[++m] = qr[i] + 1;
    }
    sort(sorted + 1, sorted + m + 1);
    int m2 = 0;
    for (int i = 1; i <= m; ++i) if (sorted[i] != sorted[i - 1]) {
      sorted[++m2] = sorted[i];
    }
    m = m2;
    for (int i = 1; i <= n; ++i) {
      ql[i] = lower_bound(sorted + 1, sorted + m + 1, ql[i]) - sorted;
      qr[i] = lower_bound(sorted + 1, sorted + m + 1, qr[i]) - sorted;
    }
    build(1, 1, m);
    int ans = 0;
    for (int i = n; i; --i) {
      int len = query(1, ql[i], qr[i]);
      if (len < qr[i] - ql[i] + 1) ans ++;
      update(1, ql[i], qr[i], 1);
    }
    printf("%lld\n", ans);
  }
}

\end{lstlisting}