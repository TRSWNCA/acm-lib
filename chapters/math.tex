\chapter{数学}

\section{快速乘法}

\begin{lstlisting}
/*
  O(1) 快速乘
*/
ll mul(ll x, ll y, ll mod) {
  return (x * y - (ll)(x / (long double)mod * y + 1e-3) * mod + mod) % mod;
}
\end{lstlisting}

\section{扩展欧几里德}

\begin{lstlisting}
/*
  扩展欧几里得
  解:ax + by = d = gcd(a, b)
  全部解:  (x0 + b/d * t, y0 - a/d * t)
*/
int exgcd(int a, int b, int &x, int &y) {
  if (b == 0) { x = 1, y = 0; return a; }
  int d = exgcd(b, a%b, y, x);
  y -= (a / b) * x; x = y; return d;
}
\end{lstlisting}

\section{中国剩余定理}

\begin{lstlisting}
/*
   中国剩余定理
   n个同余方程,第i个为x b[i](mod a[i]),要求a[i]两两互质
*/
inline void CRT(ll* a, ll* m, int n) {
  for (int i = 1; i <= n; ++ i) M *= m[i];
  for (int i = 1; i <= n; ++ i) base[i] = M / m[i];
  ll y;
  for (int i = 1; i <= n; ++ i) {
    exgcd(base[i], m[i], ans[i], y);
    (res += ans[i] * a[i] * base[i]) %= M;
  }
  printf("%lld", (res + M) % M);
}

\end{lstlisting}

\section{扩展中国定理}
\begin{lstlisting}
/*
  中国剩余定理
  n个同余方程,第i个为x a[i](mod m[i]),不要求m[i]两两互质
*/
void exCRT(ll* a, ll* b, int n) {
  ans = a[1];
  M = m[1];
  ll x, y;
  for(int i = 2; i <= n; ++i){
    ll B = ((a[i] - ans) % m[i] + m[i]) % m[i];
    ll d = exgcd(M, m[i], x, y);
    x = qmul(x, B / d, m[i]);
    ans += M * x;
    M *= m[i] / d; // lcm(M, m[i])
    ans = (ans + M) % M;
  }
  printf("%lld\n", ans);
}
\end{lstlisting}

\section{BSGS}

\begin{lstlisting}
/*
  计算满足 Y^x   Z ( mod P) 的最小非负整数
*/
int bsgs(int a, int r, int m) {
  if (r >= m) return -1;
  int i, g, x, c = 0, at = int(2 + sqrt(m));
  for (i = 0, x = 1 % m; i < 50; i++, x = ll(x) * a % m)
    if (x == r) return i;
  for (g = x = 1; __gcd(int(ll(x) * a % m), m) != g; c++)
    g = __gcd(x = ll(x) * a % m, m);
  if (r % g) return -1;
  if (x == r) return c;
  unordered_map<int, int> u;
  g = phi(m / g), u[x] = 0;
  g = qpow(a, g - at % g, m);
  for (i = 1; i < at; i++) {
    u.insert(P(x = ll(x) * a % m, i));
      if (x == r) return c + i;
  }
  for (i = 1; i < at; i++) {
    unordered_map<int, int>::iterator t = u.find(r = ll(r) * g % m);
    if (t != u.end()) return c + i * at + t->second;
  }
  return -1;
}
void solve(int y, int z, int p) {
  y %= p;
  z %= p;
  int t = bsgs(y, z, p);
  if (t == -1) {
    puts("Orz, I cannot find x!");
    return;
  } else
    printf("%d\n", t);
}
\end{lstlisting}

\section{乘法逆元}

\begin{lstlisting}
/*
  预处理对模p的逆元
*/
void pre_inv() {
  inv[1] = 1;
  for (int i = 2; i < p; ++ i) {
    int a = p / i, b = p % i;
    inv[i] = p - inv[b] * a % p;
  }
}
/*
  快速幂求逆元
*/
int get_inv(int x) {
  return qpow(x, p - 2, p);
}
/*
  阶乘预处理求模p逆元
*/
void init() {
  f[0] = 1;
  for (int i = 1; i <= p - 1; ++ i) f[i] = (ll)f[i - 1] * i % p;
  rf[p - 1] = qpow(f[p - 1], p - 2, p);
  for (int i = p - 1; i; -- i) rf[i - 1] = (ll)rf[i] * i % p;
}
\end{lstlisting}

\section{组合数}

\begin{lstlisting}
/*
  组合数, U < p
*/
namespace Comb {
const int U = 200000 + 3;
int f[U], rf[U];
ll inv(ll a, ll p) { return qpow(a, p - 2, p); }
void init() {
  f[0] = 1;
  for (int i = 1; i <= U; ++ i) f[i] = 1ll * f[i - 1] * i % p;
  rf[U] = inv(f[U], p);
  for (int i = U; i; -- i) rf[i - 1] = 1ll * rf[i] * i % p;
}
ll C(int n, int m) {
  if (m < 0 || m > n) return 0;
  return (ll)f[n] * rf[n - m] % p * rf[m];
}
} // namespace Comb
\end{lstlisting}

\section{卢卡斯定理}

\begin{lstlisting}
/*
  卢卡斯定理
  lucas(n, m, p) C(n, m) % p
*/
namespace Lucas {
ll mul(ll x, ll y, ll p) {
  return (x * y - (ll)(x / (long double)p * y + 1e-3) * p + p) % p;
}
ll qpow(ll x, ll y, ll p) {
  ll t = 1;
  for (; y; y >>= 1, x = mul(x, x, p))
  if (y & 1) t = mul(t, x, p);
  return t;
}
ll lucas(ll n, ll m, ll p) {
  if (n < m) return 0;
  if (!m || n == m) return 1;
  return Comb::C(n % p, m % p, p) * lucas(n / p, m / p, p) % p;
}
} // namespace Lucas 
\end{lstlisting}

\section{扩展卢卡斯定理}

\begin{lstlisting}
ll calc(ll n, ll x, ll P) {
  if (!n) return 1;
  ll s = 1;
  for (ll i = 1; i <= P; i++)
  if (i % x) s = s * i % P;
  s = qpow(s, n / P, P);
  for (ll i = n / P * P + 1; i <= n; i++)
  if (i % x) s = i % P * s % P;
  return s * calc(n / x, x, P) % P;
}

ll multilucas(ll n, ll m, ll x, ll P) {
  if (n < m) return 0;
  int cnt = 0;
  for (ll i = n; i; i /= x) cnt += i / x;
  for (ll i = m; i; i /= x) cnt -= i / x;
  for (ll i = n - m; i; i /= x) cnt -= i / x;
  return qpow(x, cnt, P) % P * calc(n, x, P) % P * a_1(calc(m, x, P), P) %
  P * a_1(calc(n - m, x, P), P) % P;
}

ll exlucas(ll n, ll m, ll P) {
  int cnt = 0;
  ll p[20], a[20];
  for (ll i = 2; i * i <= P; i++) {
    if (P % i == 0) {
      p[++cnt] = 1;
      while (P % i == 0) p[cnt] = p[cnt] * i, P /= i;
      a[cnt] = multilucas(n, m, i, p[cnt]);
    }
  }
  if (P > 1) p[++cnt] = P, a[cnt] = multilucas(n, m, P, P);
  return CRT(cnt, a, p);
}

\end{lstlisting}

\section{错排公式}

\begin{lstlisting}
/*
   错位排列
   求长度为n序列m位恰好在原来位置的方案数
*/
typedef long long ll;
const int N = 1000010;
const ll mod = 1e9 + 7;
ll fac[N], inv[N], f[N];
int n, m, T;
int main() {
  fac[0] = inv[0] = inv[1] = 1;
  for (int i = 1; i < N; i++) fac[i] = fac[i - 1] * i % mod;
  for (int i = 2; i < N; i++) inv[i] = (mod - mod / i) * inv[mod % i] % mod;
  for (int i = 2; i < N; i++) inv[i] = inv[i - 1] * inv[i] % mod;
  f[f[0] = f[2] = 1] = 0;
  for (int i = 3; i < N; i++) f[i] = (f[i - 1] + f[i - 2]) * (i - 1) % mod;
  scanf("%d", &T);
  while (T--) {
    scanf("%d%d", &n, &m);
    printf("%lld\n", 
      fac[n] * inv[m] % mod * inv[n - m] % mod * f[n - m] % mod);
  }
  return 0;
}
\end{lstlisting}


\section{牛顿迭代开方}

\begin{lstlisting}[language=python]
n = int(raw_input())
if n < 2:
    print(n)
    continue
m = 2
tmpn, len = n, 0
while (tmpn > 0):
    tmpn /= 10
    len += 1
base, digit, cur = 300, len / m, len % m
while (cur + m <= base) ans (digit > 0):
    cur += m
    digit -= 1
div = 10 ** (digit * m)
tmpn = n / div
x = int(float(tmpn) ** (1.0 / m))
x *= (10 ** digit)
while True:
    x0 = x
    x = x + x * (n - x ** m) / (n + m)
    if (x == x0):
        break
while (x + 1) ** m <= n:
    x = x + 1
print(x)
\end{lstlisting}

\section{二次剩余}

\begin{lstlisting}

/*
   二次剩余
   sqrt(n)%P
   Example
       (x+y)%P=b  (x*y)%P=c
       求x和y
*/
const ll P = 1e9 + 7;
const ll inv2 = (P + 1) / 2;
int T;
ll b, c;
// 求sqrt(n)%P
ll Tonelli_Shanks(ll n, ll p) {
  if (p == 2) return (n & 1) ? 1 : -1;
  if (qpow(n, p >> 1, p) + 1 == p) return -1;
  if (p & 2) return qpow(n, p + 1 >> 2, p);
  int s = __builtin_ctzll(p ^ 1);
  ll q = p >> s, z = 2;
  for (; qpow(z, p >> 1, p) == 1; ++z)
    ;
  ll c = qpow(z, q, p);
  ll r = qpow(n, q + 1 >> 1, p);
  ll t = qpow(n, q, p), tmp;
  for (int m = s, i; t != 1;) {
    for (i = 0, tmp = t; tmp != 1; ++i) tmp = tmp * tmp % p;
    for (; i < --m;) c = c * c % p;
    r = r * c % p;
    c = c * c % p;
    t = t * c % p;
  }
  return r;
}
int main() {
  scanf("%d", &T);
  while (T--) {
    scanf("%lld%lld", &b, &c);
    ll x = Tonelli_Shanks(((b * b - 4 * c) % P + P) % P, P);
    if (x == -1)
      puts("-1 -1");
    else {
      ll u1 = (b - x + P) * inv2 % P;
      ll u2 = (b + x) * inv2 % P;
      if (u1 > u2) swap(u1, u2);
      printf("%lld %lld\n", u1, u2);
    }
  }
  return 0;
}

\end{lstlisting}

\section{Pell 方程}

\begin{lstlisting}
n = int(raw_input())
j = 1
while j * j < n:
    j += 1
if j * j == n:
    print(j, 1)
if j * j > n:
    p = [0 for i in range(0, 1001)]
    q = [0 for i in range(0, 1001)]
    a = [0 for i in range(0, 1001)]
    g = [0 for i in range(0, 1001)]
    h = [0 for i in range(0, 1001)]
    p[1] = q[0] = h[1] = 1
    p[0] = q[1] = g[1] = 0
    a[2] = j - 1
    i = 2
    while 1:
        g[i] = -g[i - 1] + a[i] * h[i - 1]
        h[i] = (n - g[i] * g[i]) / h[i - 1]
        a[i + 1] = (g[i] + a[2]) / h[i]
        p[i] = a[i] * p[i - 1] + p[i - 2]
        q[i] = a[i] * q[i - 1] + q[i - 2]
        if (p[i] * p[i] - n * q[i] * q[i] == 1):
            print p[i], q[i]
            break
        i += 1

\end{lstlisting}

\section{高斯消元}

\begin{lstlisting}
/*
    高斯消元
*/
double a[20][20], b[20], c[20][20];

int main() {
  cin >> n;
  for (int i = 1; i <= n + 1; ++ i)
  for (int j = 1; j <= n; ++ j) scanf("%lf", &a[i][j]);
  for (int i = 1; i <= n; ++ i)
    for (int j = 1; j <= n; ++ j) {
      c[i][j] = 2 * (a[i][j] - a[i + 1][j]);
      b[i] += a[i][j] * a[i][j] - a[i + 1][j] * a[i + 1][j];
    }
  for (int i = 1; i <= n; ++ i) {
    for (int j = i; j <= n; ++ j) {
      if (fabs(c[j][i]) > 1e-8) {
        for (int k = 1; k <= n; ++ k) swap(c[i][k], c[j][k]);
        swap(b[i], b[j]);
        break;
      }
    }
    for (int j = 1; j <= n; ++ j) {
      if (i == j) continue;
      double rate = c[j][i] / c[i][i];
      for (int k = i; k <= n; ++ k) c[j][k] -= c[i][k] * rate;
      b[j] -= b[i] * rate;
    }
  }
  for (int i = 1; i < n; ++ i) printf("%.3f ", b[i] / c[i][i]);
  printf("%.3f\n", b[n] / c[n][n]);
}
\end{lstlisting}

\section{欧拉函数}

\begin{lstlisting}
/*
   欧拉函数
   phi(N)=N*(1-1/P1)*(1-1/P2)*...*(1-1/Pn).
   定理:
     1.欧拉函数是非完全积性函数: phi(m*n)=phi(n)*phi(m)
     当且仅当GCD(n,m)=1.
     2.一个数的所有质因子之和是Euler(n)*n/2.
     3.若n是素数p的k次幂,phi(n)=p^k-p^(k-1)=(p-1)p^(k-1),
   特殊性质:
     1.当n为奇数时,phi(2n)=phi(n).
     2.对于质数p,phi(p)=p-1
     3.除了N=2,phi(N)都是偶数.
*/
// 筛法求欧拉函数
void Get_Euler() {
  for (int i = 1; i <= n; i++) phi[i] = i;
  for (int i = 2; i <= n; i++)
    if (phi[i] == i)
      for (int j = i; j <= n; j += i) phi[j] = phi[j] / i * (i - 1);
}
int Euler(int n) {
  int t = 1, i;
  if (!(n & 1))
    for (n >>= 1; !(n & 1); n >>= 1, t <<= 1)
      ;
  for (i = 3; i * i <= n; i += 2)
    if (n % i == 0)
      for (n /= i, t *= i - 1; n % i == 0; n /= i, t *= i)
        ;
  if (n > 1) t *= n - 1;
  return t;
}
\end{lstlisting}

\section{Pollard\_Rho}

\begin{lstlisting}
/*
   Pollard_Rho
   题目大意:
   给出最大公约数和最小公倍数,满足要求的x和y,且x+y最小
   题解:
   我们发现,(x/gcd)*(y/gcd)=lcm/gcd,并且x/gcd和y/gcd互质
   那么我们先利用把所有的质数求出来Pollard_Rho,将相同的质数合并
   现在的问题转变成把合并后的质数分为两堆,使得x+y最小
   我们考虑不等式a+b>=2sqrt(ab),在a趋向于sqrt(ab)的时候a+b越小
   所以我们通过搜索求出最逼近sqrt(ab)的值即可。
 */
#define C 2730
#define S 3
using namespace std;
typedef long long ll;
ll n, m, s[1000], cnt, f[1000], cnf, ans;
ll gcd(ll a, ll b) { return b ? gcd(b, a % b) : a; }
ll mul(ll a, ll b, ll n) {
  return (a * b - (ll)(a / (long double)n * b + 1e-3) * n + n) % n;
}
ll pow(ll a, ll b, ll n) {
  ll d = 1;
  a %= n;
  while (b) {
    if (b & 1) d = mul(d, a, n);
    a = mul(a, a, n);
    b >>= 1;
  }
  return d;
}
bool check(ll a, ll n) {
  ll m = n - 1, x, y;
  int i, j = 0;
  while (!(m & 1)) m >>= 1, j++;
  x = pow(a, m, n);
  for (i = 1; i <= j; x = y, i++) {
    y = pow(x, 2, n);
    if ((y == 1) && (x != 1) && (x != n - 1)) return 1;
  }
  return y != 1;
}
bool miller_rabin(int times, ll n) {
  ll a;
  if (n == 1) return 0;
  if (n == 2) return 1;
  if (!(n & 1)) return 0;
  while (times--)
    if (check(rand() % (n - 1) + 1, n)) return 0;
  return 1;
}
ll pollard_rho(ll n, int c) {
  ll i = 1, k = 2, x = rand() % n, y = x, d;
  while (1) {
    i++, x = (mul(x, x, n) + c) % n, d = gcd(y - x, n);
    if (d > 1 && d < n) return d;
    if (y == x) return n;
    if (i == k) y = x, k <<= 1;
  }
}
void findfac(ll n, int c) {
  if (n == 1) return;
  if (miller_rabin(S, n)) {
    s[cnt++] = n;
    return;
  }
  ll m = n;
  while (m == n) m = pollard_rho(n, c--);
  findfac(m, c), findfac(n / m, c);
}
void dfs(int pos, long long x, long long k) {
  if (pos > cnf) return;
  if (x > ans && x <= k) ans = x;
  dfs(pos + 1, x, k);
  x *= f[pos];
  if (x > ans && x <= k) ans = x;
  dfs(pos + 1, x, k);
}
int main() {
  while (~scanf("%lld%lld", &m, &n)) {
    if (n == m) {
      printf("%lld %lld\n", n, n);
      continue;
    }
    cnt = 0;
    long long k = n / m;
    findfac(k, C);
    sort(s, s + cnt);
      f[0] = s[0];
    cnf = 0;
    for (int i = 1; i < cnt; i++) {
      if (s[i] == s[i - 1])
        f[cnf] *= s[i];
      else
        f[++cnf] = s[i];
    }
    long long tmp = (long long)sqrt(1.0 * k);
    ans = 1;
    dfs(0, 1, tmp);
    printf("%lld %lld\n", m * ans, k / ans * m);
  }
  return 0;
}

\end{lstlisting}

\section{FWT}

\begin{lstlisting}
/*
   快速沃尔什变化
*/
void FWT(int *a, int n) {
  for (int d = 1; d < n; d <<= 1)
    for (int m = d << 1, i = 0; i < n; i += m)
      for (int j = 0; j < d; j++) {
        int x = a[i + j], y = a[i + j + d];
        // xor:a[i+j]=x+y,a[i+j+d]=x-y;
        // and:a[i+j]=x+y;
        // or:a[i+j+d]=x+y;
      }
}
void UFWT(int *a, int n) {
  for (int d = 1; d < n; d <<= 1)
    for (int m = d << 1, i = 0; i < n; i += m)
      for (int j = 0; j < d; j++) {
        int x = a[i + j], y = a[i + j + d];
        // xor:a[i+j]=(x+y)/2,a[i+j+d]=(x-y)/2;
        // and:a[i+j]=x-y;
        // or:a[i+j+d]=y-x;
      }
}

\end{lstlisting}

\section{线性筛}

\begin{lstlisting}
/*
   mu[i]: 莫比乌斯函数
   phi[i]: 欧拉函数
   d[i]: i的质因子个数
 */
bool notp[N];
int prime[N], pnum, mu[N], c[N], d[N], phi[N];
void sieve() {
  memset(notp, 0, sizeof(notp));
  notp[0] = notp[1] = mu[1] = phi[1] = 1;
  pnum = 0;
  for (int i = 2; i < N; i++) {
    if (!notp[i]) {
      prime[++pnum] = i;
      mu[i] = -1;
      phi[i] = i - 1;
      d[i] = 1;
    }
    for (int j = 1; j <= pnum && prime[j] * i < N; j++) {
      notp[prime[j] * i] = 1;
      d[prime[j] * i] = d[i] + 1;
      if (i % prime[j] == 0) {
        mu[prime[j] * i] = 0;
        phi[i * prime[j]] = phi[i] * prime[j];
        break;
      }
      mu[prime[j] * i] = -mu[i];
      phi[i * prime[j]] = phi[i] * (prime[j] - 1);
    }
  }
}

\end{lstlisting}

\section{基础反演}

\begin{lstlisting}
/*
   f(n)= [n|d]g(d)
   g(n)= [n|d]mu[d/n]f(d)
 */
//已知 g ,求 f
for (int i = 1; i <= n; ++i)
for (int j = i + i; j <= n; j += i) f[i] += f[j];
//已知 f ,求 g
for (int i = n; i >= 1; --i)
for (int j = i + i; j <= n; j += i) f[i] -= f[j];
/*
   f(n)= [d|n]g(d)
   g(n)= [d|n]mu[n/d]f(d)= [d|n]mu[d]f(n/d)
 */
//已知 g ,求 f
for (int i = n; i >= 1; --i)
for (int j = i + i; j <= n; j += i) f[j] += f[i];
//已知 f ,求 g
for (int i = 1; i <= n; ++i)
for (int j = i + i; j <= n; j += i) f[j] -= f[i];
/*
   Example1:
   gcd(a[i],a[j],a[k],……)
   求子集gcd之和
 */
#include <bits/stdc++.h>
using namespace std;
const int N = 1010, P = 1e8 + 7;
int T, n, x, f[N], pw[N];
int main() {
  for (int i = pw[0] = 1; i < N; i++) pw[i] = pw[i - 1] * 2 % P;
  scanf("%d", &T);
  while (T--) {
    memset(f, 0, sizeof(f));
    scanf("%d", &n);
    for (int i = 1; i <= n; i++) {
      scanf("%d", &x);
      f[x]++;
    }
    n = 1000;
    for (int i = 1; i <= n; i++) {
      for (int j = i + i; j <= n; j += i) f[i] += f[j];
      f[i] = pw[f[i]] - 1;
    }
    for (int i = n; i; i--) {
      for (int j = i + i; j <= n; j += i) f[i] = (f[i] - f[j] + P) % P;
    }
    int ans = 0;
    for (int i = 1; i <= n; i++) ans = (ans + 1ll * i * f[i]) % P;
    printf("%d\n", ans);
  }
  return 0;
}
/*
   Example2:
   gcd(a[i],a[j])*(gcd(a[i],a[j])-1)
 */
#include <bits/stdc++.h>
using namespace std;
const int N = 10010, P = 10007;
int n, x, c[N], f[N];
int main() {
  while (~scanf("%d", &n)) {
    memset(c, 0, sizeof(c));
    memset(f, 0, sizeof(f));
    for (int i = 1; i <= n; i++) {
      scanf("%d", &x);
      c[x]++;
    }
    n = 10000;
    for (int i = 1; i <= n; i++) {
      for (int j = i; j <= n; j += i) f[i] += c[j];
      f[i] = f[i] * f[i] % P;
    }
    for (int i = n; i >= 1; --i)
      for (int j = i + i; j <= n; j += i) f[i] -= f[j];
    int ans = 0;
    for (int i = 2; i <= n; i++)
      ans = (ans + f[i] * i % P * (i - 1) % P) % P;
    printf("%d\n", ans);
  }
  return 0;
}
/*
   Example3:
   gcd(a[i],a[j],a[k],a[l])
 */
#include <cstdio>
#include <cstring>
using namespace std;
const int N = 10010;
int n, x, c[N];
long long f[N];
int main() {
  while (~scanf("%d", &n)) {
    memset(c, 0, sizeof(c));
    memset(f, 0, sizeof(f));
    for (int i = 1; i <= n; i++) {
      scanf("%d", &x);
      c[x]++;
    }
    n = 10000;
    for (int i = 1; i <= n; i++) {
      for (int j = i; j <= n; j += i) f[i] += c[j];
        if (f[i] > 3)
          f[i] = f[i] * (f[i] - 1) * (f[i] - 2) * (f[i] - 3) / 24;
        else
          f[i] = 0;
    }
    for (int i = n; i >= 1; --i)
      for (int j = i + i; j <= n; j += i) f[i] -= f[j];
    printf("%lld\n", f[1]);
  }
  return 0;
}
\end{lstlisting}

\section{莫比乌斯反演}
\begin{lstlisting}
/*
   莫比乌斯反演
   f(n)= [n|d]g(d)
   g(n)= [n|d]mu[d/n]f(d)
   ----------------------
   f(n)= [d|n]g(d)
   g(n)= [d|n]mu[n/d]f(d)= [d|n]mu[d]f(n/d)
 */
/*
   Example1:
   (i,a,b) (j,c,d)[gcd(i,j)==k]
 */
#include <bits/stdc++.h>
using namespace std;
typedef long long ll;
const int N = 100010;
bool notp[N];
int prime[N], pnum, mu[N];
void sieve() {
  memset(notp, 0, sizeof(notp));
  notp[0] = notp[1] = mu[1] = 1;
  for (int i = 2; i < N; i++) {
    if (!notp[i]) {
      prime[++pnum] = i;
      mu[i] = -1;
    }
    for (int j = 1; j <= pnum && prime[j] * i < N; j++) {
      notp[prime[j] * i] = 1;
      if (i % prime[j] == 0) {
        mu[prime[j] * i] = 0;
        break;
      }
      mu[prime[j] * i] = -mu[i];
    }
  }
}
ll f[N];
ll solve(int n, int m) {
  if (n > m) swap(n, m);
  ll res = 0;
  for (int i = 1, last; i <= n; i = last + 1) {
    last = min(n / (n / i), m / (m / i));
    res += (f[last] - f[i - 1]) * (n / i) * (m / i);
  }
  return res;
}
int T, a, b, c, d, k;
int main() {
  sieve();
  for (int i = 1; i < N; i++) f[i] = f[i - 1] + mu[i];
  scanf("%d", &T);
  while (T--) {
    scanf("%d%d%d%d%d", &a, &b, &c, &d, &k);
    printf("%lld\n", solve(b / k, d / k) - solve(b / k, (c - 1) / k) -
        solve((a - 1) / k, d / k) +
        solve((a - 1) / k, (c - 1) / k));
  }
  return 0;
}
/*
   Example2:
   [gcd(i,j)素因子个数<=P]
 */
#include <bits/stdc++.h>
using namespace std;
using namespace std;
typedef long long ll;
const int N = 500010;
bool notp[N];
int prime[N], pnum, mu[N], d[N];
void sieve() {
  memset(notp, 0, sizeof(notp));
  notp[0] = notp[1] = mu[1] = 1;
  for (int i = 2; i < N; i++) {
    if (!notp[i]) {
      prime[++pnum] = i;
      mu[i] = -1;
        d[i] = 1;
    }
    for (int j = 1; j <= pnum && prime[j] * i < N; j++) {
      notp[prime[j] * i] = 1;
      d[prime[j] * i] = d[i] + 1;
      if (i % prime[j] == 0) {
        mu[prime[j] * i] = 0;
        break;
      }
      mu[prime[j] * i] = -mu[i];
    }
  }
}
int f[N][21];
void calF() {
  for (int i = 1; i < N; i++)
    for (int j = i; j < N; j += i) f[j][d[i]] += mu[j / i];
  for (int i = 1; i < N; i++)
    for (int j = 1; j <= 20; j++) f[i][j] += f[i][j - 1];
  for (int i = 1; i < N; i++)
    for (int j = 0; j <= 20; j++) f[i][j] += f[i - 1][j];
}
ll solve(int n, int m, int p) {
  if (n > m) swap(n, m);
  ll res = 0;
  for (int i = 1, last; i <= n; i = last + 1) {
    last = min(n / (n / i), m / (m / i));
    res += (ll)(f[last][p] - f[i - 1][p]) * (n / i) * (m / i);
  }
  return res;
}
int T, n, m, p;
int main() {
  sieve();
  calF();
  scanf("%d", &T);
  while (T--) {
    scanf("%d%d%d", &n, &m, &p);
    p = min(p, 20);
    printf("%lld\n", solve(n, m, p));
  }
  return 0;
}

\end{lstlisting}

\section{杜教筛}
\begin{lstlisting}

/*
   杜教筛
   g(1)S(n)=\sum_{i=1}^{n}h(i)−\sum_{d=2}^{n}g(d)S(n/d)
 */
/*
   Example1:
   莫比乌斯函数区段和
 */
const int mod = 1333331;
typedef long long LL;
LL a, b, miu[5000010];
int p[500010], cnt = 0;
bool vis[5000010];
struct HASHMAP {
  int h[mod + 10], cnt, nxt[100010];
  LL st[100010], S[100010];
  void push(LL k, LL v) {
    int key = k % mod;
    for (int i = h[key]; i; i = nxt[i]) {
      if (S[i] == k) return;
    }
    ++cnt;
    nxt[cnt] = h[key];
    h[key] = cnt;
    S[cnt] = k;
    st[cnt] = v;
  }
  LL ask(LL k) {
    int key = k % mod;
    for (int i = h[key]; i; i = nxt[i]) {
      if (S[i] == k) return st[i];
    }
    return -1;
  }
} H;
void Get_Prime() {
  miu[1] = 1;
  for (int i = 2; i <= 5000000; ++i) {
    if (!vis[i]) {
      p[++cnt] = i;
      miu[i] = -1;
    }
    for (int j = 1; j <= cnt; ++j) {
      if (1LL * p[j] * i > 5000000) break;
      int ip = i * p[j];
      vis[ip] = true;
      if (i % p[j] == 0) break;
      miu[ip] = -miu[i];
    }
  }
  for (int i = 2; i <= 5000000; ++i) miu[i] += miu[i - 1];
}
LL miu_sum(LL n) {
  if (n <= 5000000) return miu[n];
  LL tmp = H.ask(n), la, A = 1;
  if (tmp != -1) return tmp;
  for (LL i = 2; i <= n; i = la + 1) {
    LL now = n / i;
    la = n / now;
    A = A - (la - i + 1) * miu_sum(n / i);
  }
  H.push(n, A);
  return A;
}
int main() {
  scanf("%lld%lld", &a, &b);
  Get_Prime();
  printf("%lld\n", miu_sum(b) - miu_sum(a - 1));
  return 0;
}
/*
   Example2:
   欧拉函数前缀和
 */
typedef long long LL;
const int mod = 1333331, inv2 = 500000004;
const LL MOD = 1e9 + 7;
LL a, b, miu[5000010], phi[5000010];
int p[500010], cnt = 0, i, tot;
bool v[5000010];
struct HASHMAP {
  int h[mod + 10], cnt, nxt[100010];
  LL st[100010], S[100010];
  void push(LL k, LL v) {
    int key = k % mod;
    for (int i = h[key]; i; i = nxt[i]) {
      if (S[i] == k) return;
    }
    ++cnt;
    nxt[cnt] = h[key];
    h[key] = cnt;
    S[cnt] = k;
    st[cnt] = v;
  }
  LL ask(LL k) {
    int key = k % mod;
    for (int i = h[key]; i; i = nxt[i]) {
      if (S[i] == k) return st[i];
    }
    return -1;
  }
} H;
void Get_Prime() {
  for (miu[1] = phi[1] = 1, i = 2; i <= 5000000; i++) {
    if (!v[i]) p[tot++] = i, miu[i] = -1, phi[i] = i - 1;
    for (int j = 0; i * p[j] <= 5000000 && j < tot; j++) {
      v[i * p[j]] = 1;
      if (i % p[j]) {
        miu[i * p[j]] = -miu[i];
        phi[i * p[j]] = phi[i] * (p[j] - 1);
      } else {
        miu[i * p[j]] = 0;
        phi[i * p[j]] = phi[i] * p[j];
        break;
      }
    }
  }
  for (int i = 2; i <= 5000000; ++i) phi[i] = (phi[i - 1] + phi[i]) % MOD;
}
LL phi_sum(LL n) {
  if (n <= 5000000) return phi[n];
  LL tmp = H.ask(n), la, A = 0;
  if (tmp != -1) return tmp;
  for (LL i = 2; i <= n; i = la + 1) {
    LL now = n / i;
    la = n / now;
    (A += (la - i + 1) % MOD * phi_sum(n / i) % MOD) %= MOD;
  }
  A = ((n % MOD) * (n % MOD + 1) % MOD * inv2 % MOD - A + MOD) % MOD;
  H.push(n, A);
  return A;
}
int main() {
  scanf("%lld", &a);
  Get_Prime();
  printf("%lld\n", phi_sum(a));
  return 0;
}
/*
   Example3:
   求[1,n][1,n]最大公约数之和
   枚举最大公约数k,答案为2* (k*phi_sum(n/k))-n*(n+1)/2
   phi_sum利用杜教筛实现
 */
typedef long long LL;
const int mod = 1333331, inv2 = 500000004;
const LL MOD = 1e9 + 7;
LL a, b, n, miu[5000010], phi[5000010];
int p[500010], cnt = 0, i, tot;
bool v[5000010];
struct HASHMAP {
  int h[mod + 10], cnt, nxt[100010];
  LL st[100010], S[100010];
  void push(LL k, LL v) {
    int key = k % mod;
    for (int i = h[key]; i; i = nxt[i]) {
      if (S[i] == k) return;
    }
    ++cnt;
    nxt[cnt] = h[key];
    h[key] = cnt;
    S[cnt] = k;
      st[cnt] = v;
  }
  LL ask(LL k) {
    int key = k % mod;
    for (int i = h[key]; i; i = nxt[i]) {
      if (S[i] == k) return st[i];
    }
    return -1;
  }
} H;
void Get_Prime() {
  for (miu[1] = phi[1] = 1, i = 2; i <= 5000000; i++) {
    if (!v[i]) p[tot++] = i, miu[i] = -1, phi[i] = i - 1;
    for (int j = 0; i * p[j] <= 5000000 && j < tot; j++) {
      v[i * p[j]] = 1;
      if (i % p[j]) {
        miu[i * p[j]] = -miu[i];
        phi[i * p[j]] = phi[i] * (p[j] - 1);
      } else {
        miu[i * p[j]] = 0;
        phi[i * p[j]] = phi[i] * p[j];
        break;
      }
    }
  }
  for (int i = 2; i <= 5000000; ++i) phi[i] = (phi[i - 1] + phi[i]) % MOD;
}
LL phi_sum(LL n) {
  if (n <= 5000000) return phi[n];
  LL tmp = H.ask(n), la, A = 0;
  if (tmp != -1) return tmp;
  for (LL i = 2; i <= n; i = la + 1) {
    LL now = n / i;
    la = n / now;
    (A += (la - i + 1) % MOD * phi_sum(n / i) % MOD) %= MOD;
  }
  A = ((n % MOD) * (n % MOD + 1) % MOD * inv2 % MOD - A + MOD) % MOD;
  H.push(n, A);
  return A;
}
int main() {
  scanf("%lld", &n);
  Get_Prime();
  LL la, ans = 0;
  for (LL i = 1; i <= n; i = la + 1) {
    LL now = n / i;
    la = n / now;
    ans = (ans + (i + la) % MOD * (la - i + 1) % MOD * inv2 % MOD *
        phi_sum(now) % MOD) %
      MOD;
  }
  ans = ans * 2 % MOD;
  LL k = n % MOD;
  ans = (ans - k * (k + 1) % MOD * inv2 % MOD + MOD) % MOD;
  printf("%lld\n", ans);
  return 0;
}
/*
   Example4:
   求[1,n][1,m]两两LCM的和
 */
namespace Sum_LCM {
  int tot, p[N], miu[N], v[N];
  LL sum[N];
  void mobius(int n) {
    int i, j;
    for (miu[1] = 1, i = 2; i <= n; i++) {
      if (!v[i]) p[tot++] = i, miu[i] = -1;
      for (j = 0; j < tot && i * p[j] <= n; j++) {
        v[i * p[j]] = 1;
        if (i % p[j])
          miu[i * p[j]] = -miu[i];
        else
          break;
      }
    }
    for (i = 1; i <= n; i++)
      sum[i] = (sum[i - 1] + (1LL * miu[i] * i % mod * i % mod)) % mod;
  }
  LL Sum(LL x, LL y) {
    return (((x * (x + 1) / 2) % mod) * ((y * (y + 1) / 2) % mod) % mod);
  }
  LL F(int n, int m) {
    LL t = 0;
    if (n > m) swap(n, m);
    for (int i = 1, j = 0; i <= n; i = j + 1) {
      j = min(n / (n / i), m / (m / i));
      t = (t + (sum[j] - sum[i - 1] + mod) % mod * Sum((LL)n / i, (LL)m / i) %
          mod) %
        mod;
    }
    return t;
  }
  LL Cal_Sum(int n, int m) {
    if (n > m) swap(n, m);
    mobius(n);
    LL ans = 0;
    for (int i = 1, j = 0; i <= n; i = j + 1) {
      j = min(n / (n / i), m / (m / i));
      ans = (ans +
          1LL * (i + j) * (j - i + 1) / 2 % mod * F(n / i, m / i) % mod) %
        mod;
    }
    return ans;
  }
} // namespace Sum_LCM

\end{lstlisting}


\section{Min25 筛}

\begin{lstlisting}

/*
   Min25筛
   S(x,y):  F(i)[i<=x&&i的最小质因子不小于P_y]
   pi>n时S(n,i)=0
   pi<=n时S(n,i)=\sum_{j>=i}^{(p_j)^2<=n}\sum_{c=1}^{(p_j)^{c+1}}S(n/p_i,j+1)*F(p_i^c)+F(p_i^{c+1})
   +\sum_{x是质数且x<=n}F(x)-\sum_{j=1}^{i-1}F(p_i)
   F(i)的前缀和答案为S(n,1)+F(1)
 */
/*
   Example1:
   欧拉函数前缀和&&莫比乌斯函数前缀和
 */
#include <bits/stdc++.h>
typedef long long ll;
const int N = 500000 << 1; // sqrt(n)*2
int n, m, Sqr, cnt, P[N >> 2], h[N], w[N];
ll sp[N], g[N];
inline int ID(int x) { return x <= Sqr ? x : m - n / x + 1; }
// g:前缀质数和
// h:前缀质数个数
void Init(int n) {
  Sqr = sqrt(n);
  cnt = m = 0;
  for (ll i = 1; i <= n; i = w[m] + 1ll)
    w[++m] = n / (n / i), g[m] = (1ll * w[m] * ((ll)w[m] + 1) >> 1) - 1,
      h[m] = w[m] - 1;
  for (int i = 2; i <= Sqr; ++i)
    if (h[i] != h[i - 1]) {
      P[++cnt] = i, sp[cnt] = sp[cnt - 1] + i;
      int lim = i * i;
      for (int j = m; lim <= w[j]; --j) {
        int k = ID(w[j] / i);
        g[j] -= 1ll * i * (g[k] - sp[cnt - 1]);
        h[j] -= h[k] - cnt + 1;
      }
    }
}
// 欧拉函数前缀和
ll S_Phi(int x, int y) {
  if (x <= 1 || P[y] > x) return 0;
  ll res = g[ID(x)] - h[ID(x)] - sp[y - 1] + y - 1; // g-h
  for (int i = y; i <= cnt && P[i] * P[i] <= x; ++i)
    for (ll p = P[i], p1 = p, t = p - 1; 1ll * p1 * p <= x; p1 *= p, t *= p)
      res += 1ll * (S_Phi(x / p1, i + 1) + p) * t;
  return res;
}
// 莫比乌斯函数前缀和
int S_Mu(int x, int y) {
  if (x <= 1 || P[y] > x) return 0;
  int res = -h[ID(x)] + y - 1; // h
  for (int i = y; i <= cnt && P[i] * P[i] <= x; ++i)
    res -= S_Mu(x / P[i], i + 1);
  return res;
}
int main() {
  int T;
  scanf("%d", &T);
  while (T--) {
    scanf("%d", &n);
    if (!n)
      puts("0 0");
    else {
      Init(n);
      printf("%lld %d\n", S_Phi(n, 1) + 1, S_Mu(n, 1) + 1);
    }
  }
  return 0;
}
/*
    Example2:
        F(1)=1
        F(p^c)=p xor c(p的c次的函数值为p和c的异或值)
        F(ab)=F(a)F(b) gcd(a,b)==1
        求F前缀和(n<=10^10)
    Solution:
        除了2之外,p xor 1等于p-1,
        因此我们可以用质数常函数前缀g和质数前缀个数函数h
        来计算 _{x是质数且x<=n}F(x)
 */
#include <bits/stdc++.h>
typedef long long ll;
const int N = 100000 << 1 | 1; // sqrt(n)*2
const int P = 1e9 + 7;
int m, cnt, pri[N >> 2], h[N];
ll n, Sqr, sp[N], g[N], w[N];
inline int ID(ll x) { return x <= Sqr ? x : m - n / x + 1; }
// g:前缀质数和
// h:前缀质数个数
void Init(ll n) {
  Sqr = sqrt(n);
  cnt = m = 0;
  for (ll i = 1; i <= n; i = w[m] + 1) {
    w[++m] = n / (n / i);
    ll t = w[m] % P;
    g[m] = (t * (t + 1) >> 1) % P - 1;
    h[m] = w[m] - 1;
  }
  for (int i = 2; i <= Sqr; ++i){
    if (h[i] != h[i - 1]) {
      pri[++cnt] = i, sp[cnt] = sp[cnt - 1] + i;
      ll lim = 1ll * i * i;
      for (int j = m; lim <= w[j]; --j) {
        int k = ID(w[j] / i);
        g[j] -= i * (g[k] - sp[cnt - 1]);
        h[j] -= h[k] - cnt + 1;
      }
    }
    g[i] %= P;
  }
}
ll S(ll x, int y) {
  if (x <= 1 || pri[y] > x) return 0;
  ll res = g[ID(x)] - g[ID(pri[y - 1])] + P;
  for (int i = y; i <= cnt && 1ll * pri[i] * pri[i] <= x; ++i) {
    for (ll p = pri[i], p1 = p, j = 1; p1 * p <= x; p1 *= p, ++j)
      res += S(x / p1, i + 1) * (pri[i] ^ j) + (pri[i] ^ (j + 1));
  }
  return res % P;
}
int main() {
    scanf("%lld", &n);
  Init(n);
  for (int i = 1; i <= m; ++i) g[i] = (g[i] - h[i] + (i > 1) * 2 + P) % P;
  printf("%lld\n", (S(n, 1) + 1) % P);
  return 0;
}
/*
    Example3:
        对任意p为素数,
        p%4==1时有f(p^c)=3*c+1
        p%4!=1时f(p^c)=1
        对于非质数有
        f(a*b)=f(a)*f(b) gcd(a,b)==1
    Solution:
        处理g[n][r]表示前缀%4==r的质数个数,可以通过Min25预处理
        设G[n]等于4*g[n][1]+g[n][3],对于唯一素数2特殊处理,n>1时G[n]++
        G[n]= F(x)_x为小于n的质数,G[pri[i-1]]= _{j=1}^{i-1}F(p_i)
        套Min25得到f前缀和即可
 */
#include <bits/stdc++.h>
typedef long long ll;
const int N = 50000 << 1 | 1; // sqrt(n)*2
int n, m, Sqr, pnum, pri[N >> 2], w[N];
ll g[N][4], sp[N][4], G[N];
inline int ID(int x) { return x <= Sqr ? x : m - n / x + 1; }
int F(int p, int k) {
  if (p % 4 == 1) return 3 * k + 1;
  return 1;
}
bool notp[N];
void sieve() {
  memset(notp, 0, sizeof(notp));
  notp[0] = notp[1] = 1;
  pnum = 0;
  for (int i = 2; i < N; i++) {
    if (!notp[i]) {
      pri[++pnum] = i;
      for (int r = 0; r < 4; ++r)
        sp[pnum][r] = sp[pnum - 1][r] + (i % 4 == r);
    }
    for (int j = 1; j <= pnum && pri[j] * i < N; j++) {
      notp[pri[j] * i] = 1;
      if (i % pri[j] == 0) break;
    }
  }
}

\end{lstlisting}

\section{约瑟夫问题}

\begin{lstlisting}
/*
   约瑟夫问题
 */
int Josephus(int n, int k) { // Number(0-n-1)
  if (k == 1) return n - 1;
  if (n == 1) return 0;
  int ret;
  if (n < k) {
    ret = 0;
    for (int i = 2; i <= n; i++) ret = (ret + k) % i;
    return ret;
  }
  ret = Josephus(n - n / k, k);
  if (ret < n % k)
      ret = ret - n % k + n;
  else
    ret = ret - n % k + (ret - n % k) / (k - 1);
  return ret;
}

\end{lstlisting}

\section{分数类}

\begin{lstlisting}
/*
   分数类
 */
const int N = 500;
struct fraction {
  long long numerator; // 分子
  long long denominator; // 分母
  fraction() {
    numerator = 0;
    denominator = 1;
  }
  fraction(long long num) {
    numerator = num;
    denominator = 1;
  }
  fraction(long long a, long long b) {
    numerator = a;
    denominator = b;
    this->reduction();
  }
  void operator=(const long long num) {
    numerator = num;
    denominator = 1;
    this->reduction();
  }
  void operator=(const fraction &b) {
    numerator = b.numerator;
    denominator = b.denominator;
    this->reduction();
  }
  fraction operator+(const fraction &b) const {
    long long gcdnum = __gcd(denominator, b.denominator);
    fraction tmp = fraction(numerator * (b.denominator / gcdnum) +
        b.numerator * (denominator / gcdnum),
        denominator / gcdnum * b.denominator);
    tmp.reduction();
    return tmp;
  }
  fraction operator+(const int &b) const { return ((*this) + fraction(b)); }
  fraction operator-(const fraction &b) const {
    return ((*this) + fraction(-b.numerator, b.denominator));
  }
  fraction operator-(const int &b) const { return ((*this) - fraction(b)); }
  fraction operator*(const fraction &b) const {
    fraction tmp =
      fraction(numerator * b.numerator, denominator * b.denominator);
    tmp.reduction();
    return tmp;
  }
  fraction operator*(const int &b) const { return ((*this) * fraction(b)); }
  fraction operator/(const fraction &b) const {
    return ((*this) * fraction(b.denominator, b.numerator));
  }
  void reduction() {
    if (numerator == 0) {
      denominator = 1;
      return;
    }
    long long gcdnum = __gcd(numerator, denominator);
    numerator /= gcdnum;
    denominator /= gcdnum;
  }
  void print() {
    if (denominator == 1)
      printf("%lld\n", numerator);
    else {
      long long num = numerator / denominator;
      long long tmp = num;
      int len = 0;
      while (tmp) {
        len++;
        tmp /= 10;
      }
      for (int i = 0; i < len; i++) printf(" ");
      if (len != 0) printf(" ");
      printf("%lld\n", numerator % denominator);
      if (num != 0) printf("%lld ", num);
      tmp = denominator;
      while (tmp) {
        printf("-");
        tmp /= 10;
      }
      puts("");
      for (int i = 0; i < len; i++) printf(" ");
      if (len != 0) printf(" ");
      printf("%lld\n", denominator);
    }
  }
} f[N];
int n;
/*
   要抽齐n种卡片期望要买的卡片数
 */
int main() {
  while (~scanf("%d", &n)) {
    f[0] = 0;
    for (int i = 1; i <= n; i++) f[i] = f[i - 1] + fraction(1, i);
    f[n] = f[n] * n;
    f[n].print();
  }
  return 0;
}

\end{lstlisting}
